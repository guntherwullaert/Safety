\documentclass{article}
\usepackage{amsmath}
\usepackage{comments}
\usepackage{hyperref}

\title{Definition of Safety}
\author{Günther Wullaert}
\date{May 2022}

\newcommand{\pool}[1]{\boldsymbol{#1}}
\newcommand{\tuple}[1]{\dot{#1}}

\newcommand{\set}[1]{\{#1\}}
\newcommand{\dep}[2]{\{(#1), (#2)\}}
\newcommand{\provide}[2]{\{(\{#1\}, \{#2\})\}}
\newcommand{\provideM}[4]{\{(\{#1\}, \{#2\}), (\{#3\},\{#4\})\}}
\newcommand{\provideMM}[6]{\{(\{#1\}, \{#2\}), (\{#3\},\{#4\}), (\{#5\},\{#6\})\}}
\newcommand{\provideMMM}[8]{\{(\{#1\}, \{#2\}), (\{#3\},\{#4\}), (\{#5\},\{#6\}), (\{#7\},\{#8\})\}}

\newcommand\Vars{\mathit{vars}}
\newcommand\Eval{\mathit{eval}}
\newcommand\Negate{\mathit{negate}}
\newcommand\Provide{\mathit{pt}}
\newcommand\Depend{\mathit{dt}}
\newcommand\ProDep{\mathit{dep}}
\newcommand\Analyze{\mathit{analyze}}
\newcommand\PD{\mathit{P\!D}}
\newcommand\CheckOp[1]{C_{#1}}

\begin{document}
	\maketitle

	\section{Language}
	We define our language over four sets of symbols:
	numerals, symbolic constants, variables, and aggregate names.


	\subsection{Term and Pools}

	We inductively define \emph{terms}, \emph{tuples of terms}, and \emph{pools}:
	\begin{itemize}
		\item all numerals and variables are terms,
		\item $f(\pool{t})$ is a term if $f$ is a symbolic constant and $\pool{t}$ is a pool,
		\item $t_1 \star t_2$ is a term if $t_1$, $t_2$ are terms and ${\star} \in \set{{+}, {-}, {\times}, {\div}, {..}}$,
		\item $\langle \pool{t} \rangle$ is a term if $\pool{t}$ is a pool,
		\item $t_1,\dots,t_n$ is a tuple of terms if $n \ge 0$ and $t_i$ is a term,
		\item $\tuple{t_1};\dots;\tuple{t_n}$ is a pool if $n \ge 1$ and each $\tuple{t_i}$ is a tuple of terms.
	\end{itemize}
	We omit writing the parenthesis for terms of form~\(f()\).

	We inductively define a term to be $\emph{evaluable}$ if
	\begin{itemize}
		\item it is a numeral, or
		\item it has form $t_1 \star t_2$ where $t_1$ and $t_2$ are evaluable and $\star \in \set{{+}, {-}, {\times}, {\div}}$.
	\end{itemize}
	%
	We then inductively define function $\Eval$ mapping evaluable terms to sets of numerals:
	\begin{itemize}
	\item for numerals~\(t\), we let $\Eval(t) = \{t\}$, and
	\item for terms of form $t_1 \star t_2$, we let
	\begin{alignat*}{4}
		\Eval(t_1 + t_2) &= \{ s_1 + s_2 &&\mid s_1 \in \Eval(t_1), s_2 \in \Eval(t_2) \}, \\
		\Eval(t_1 - t_2) &= \{ s_1 - s_2 &&\mid s_1 \in \Eval(t_1), s_2 \in \Eval(t_2) \}, \\
		\Eval(t_1 \times t_2) &= \{ s_1 \times s_2 &&\mid s_1 \in \Eval(t_1), s_2 \in \Eval(t_2) \}\text{, and} \\
		\Eval(t_1 \div t_2) &= \{ s_1 \div s_2 &&\mid s_1 \in \Eval(t_1), s_2 \in \Eval(t_2), s_2 \neq 0 \}.
	\end{alignat*}
	\end{itemize}
	We say that a term~$t$ is \emph{nonzero} if it is evaluable and $0 \notin \Eval(t)$.

	\subsection{Atoms and Literals}

	An \emph{atom} has form $p(\pool{t})$ where $p$ is a symbolic constant and $\pool{t}$ is a pool.
	We omit writing parenthesis for atoms of form~\(p()\).

	A \emph{comparison} has form $t_1 \prec t_2$, where $t_1,t_2$ are terms and ${\prec} \in \set{{<}, {>}, {\leq}, {\geq}, {=}, {\neq}}$.
	Furthermore, we let $\Negate$ be a function to negate relation symbols:
	${<} \mapsto {\geq}$,
	${>} \mapsto {\leq}$,
	${\leq} \mapsto {>}$,
	${\geq} \mapsto {<}$,
	${=} \mapsto {\neq}$ and
	${\neq} \mapsto {=}$.

	A \emph{literal} is either an atom or a comparison optionally preceded by the \emph{default negation} symbol $\neg$.

	A \emph{conditional literal} has form $l : \tuple{l}$, where $l$ is a literal and $\tuple{l}$ is a (possibly empty) tuple of literals.

	\subsection{Aggregates}
	An \emph{aggregate} has the form
	\begin{align}
		&\alpha\{\tuple{t_1} : \tuple{l_1}; \dots; \tuple{t_n} : \tuple{l_n}\} \prec s \label{aggregate}
	\end{align}
	where
	\begin{itemize}
		\item $n \geq 0$,
		\item $\alpha$ is an aggregate name,
		\item each $\tuple{t_i}$ is a tuple of terms,
		\item each $\tuple{l_i}$ is a (possible empty) tuple of literals,
		\item ${\prec} \in \set{{<}, {>}, {\leq}, {\geq}, {=}, {\neq}}$, and
		\item $s$ is a term.
	\end{itemize}

	\subsection{Rules}

	A \emph{rule} has form
	\begin{align}
		a_1 \vee \dots \vee a_m &\leftarrow l_1 \wedge \dots \wedge l_n \label{rule}
	\end{align}
	where $m, n \ge 0$, each~$a_i$ is an atom and each~$l_i$ is a literal, conditional literal or aggregate.

	A \emph{choice rule} has form
	\begin{align}
		\{a_1; \dots; a_m\} &\leftarrow l_1 \wedge \dots \wedge l_n \label{choice}
	\end{align}
	where
	$m, n \geq 0$,
	each~$a_i$ is an atom and
	each~$l_i$ is a literal, conditional literal or aggregate.

	We refer to the $a_i$ and $l_i$ in rules of form~\eqref{rule} and~\eqref{choice} as \emph{head atoms} and \emph{body literals}, respectively.

	\section{Safety}

	In the following, we use function $\Vars(e)$ to obtain all variables occurring in an expression $e$.
  %
	Furthermore, we say that a variable $X$ occurs \emph{globally} in
	\begin{itemize}
		\item a literal~$l$ if $X \in \Vars(l)$,
		\item a conditional literal~$l : \tuple{l}$ if $X \in \Vars(l) \setminus \Vars(\tuple{l})$,
		\item an aggregate of form~\eqref{aggregate} if $X \in s$, and
		\item a rule of form~\eqref{rule} or~\eqref{choice} if it occurs globally in a head atom or body literal.
	\end{itemize}

	\subsection{Terms}

	We inductively define function~$\Provide$ for terms, tuples of terms and pools:
	\begin{itemize}
		\item
			for numerals~$n$, we let
			$\Provide(n) = \emptyset$ and $\Depend(n) = \emptyset$,
		\item
			for variables~$X$, we let
			$\Provide(X) = \set{X}$ and $\Depend(X) = \emptyset$,
		\item
			for term tuples~$\tuple{t} = t_1,\dots,t_n$, we let
			\begin{align*}
				\Provide(\tuple{t}) &= \Provide(t_1) \cup \dots \cup \Provide(t_n)\text{\ and} \\
				\Depend(\tuple{t}) &= (\Depend(t_1) \cup \dots \cup \Depend(t_n)) \setminus \Provide(\tuple{t}),
			\end{align*}
		\item
			for pools~$\pool{t} = \tuple{t_1};\dots;\tuple{t_n}$, we let
			\begin{align*}
				\Provide(\pool{t}) &= \Provide(\tuple{t_1}) \cap \dots \cap \Provide(\tuple{t_n})\text{\ and} \\
				\Depend(\pool{t}) &= (\Depend(t_1) \cup \dots \cup \Depend(t_n)) \setminus \Provide(\pool{t}),
			\end{align*}
		\item
			for terms of form~$f(\pool{t})$, we let
			$\Provide(f(\pool{t})) = \Provide(\pool{t})$ and $\Depend(f(\pool{t})) = \Depend(\pool{t})$,
		\item
			for terms of form~$t_1 \star t_2$, we let
				\begin{align*}
				\Provide(t_1 \star t_2) &= \begin{cases}
					\Provide(t_2) & \text{$t_1$ is evaluable and ${\star} \in \set{{+}, {-}}$}, or\\
					              & \text{$t_1$ is nonzero and ${\star} = {\times}$,}\\
					\Provide(t_1) & \text{$t_2$ is evaluable and ${\star} \in \set{{+}, {-}}$}, or\\
					              & \text{$t_2$ is nonzero and ${\star} = {\times}$,}\\
					\emptyset & \text{otherwise}
				\end{cases}\\
				\Depend(\pool{t}) &= vars(t_1 \star t_2) \setminus \Provide(t_1 \star t_2)
				\end{align*}
	\end{itemize}

	\subsection{Body Literals}
	Next, we define function~$\ProDep$ for body literals:
	\begin{itemize}
		\item 
			for an atom~$a$ of form~$p(\pool{t})$, we let
			$\ProDep(a) = \set{(\Provide(\pool{t}), \emptyset), (\emptyset, \Depend(\pool{t}))}$,
		\item
			for a comparison~$a$ of form $t_1 \prec t_2$ with ${\prec} \notin \set{{=}}$, we let
			$\ProDep(a) = \set{(\emptyset, \Vars(a))}$,
		\item
			for a comparison~$a$ of form $t_1 = t_2$, we let
			$\ProDep(a) = \set{(\Provide(t_1), \Vars(t_2)), (\Provide(t_2), \Vars(t_1)), (\emptyset, \Depend(t_1) \cup \Depend(t_2))}$,
		\item 
			for a literal $l$ of form $\neg a$ where $a$ is an atom, we let
			$\ProDep(l) = \set{(\emptyset, \Vars(l))}$,
		\item 
			for a literal $l$ of form $\neg t_1 \prec t_2$, we let
			$\ProDep(l) = \ProDep(t_1 \mathrel{\Negate({\prec})} t_2)$,
		\item
			for conditional literal $l$, we let
			$\ProDep(l) = \set{(\emptyset, \Vars(l))}$,\comment{%
				We could use $\ProDep(l) = \set{(\emptyset, \Depend(l_0) \cup \Vars(l_1,\dots,l_n))}$
				but would have to check that the remaining variables in $l_0$ are either provided elsewhere or do not occur in other conditional literals.
				For example, \textit{clingo} accepts\par  $\quad\leftarrow p(X,Y):q(Y)$\par and\par $\quad\leftarrow p(X,Y):q(Y) \wedge r(X;Z)$\par as safe
				but should discard\par $\quad\leftarrow p(X,Y):q(Y) \wedge r(X,Y):s(Y)$\par as unsafe.\par
				\quad On the other hand, we could also omit this case altogether because it should not be relevant in practice at all.
				(In \textit{clingo}, it is only implemented because of an internal rewriting step.)
			}
		\item
			for an aggregate~$l$ of form~\eqref{aggregate} with ${\prec} \notin \set{{=}}$, we let
			$\ProDep(l) = \set{(\emptyset, \Vars(l))}$,
		\item
			for an aggregate~$l$ of form~\eqref{aggregate} with ${\prec} \in \set{{=}}$, we let
			$\ProDep(l) = \set{(\Provide(s), \Vars(\tuple{t_1} : \tuple{l_1}; \dots; \tuple{t_n} : \tuple{l_n})), (\emptyset,\Depend(s))}$.
	\end{itemize}

\subsection{Gathering Dependencies}
	Next, we define function~$\Analyze$ to gather dependencies in rules, conditional literals, and aggregate elements:
	\begin{itemize}
		\item
			for a rule $r$ of form~\eqref{rule} or~\eqref{choice}, we let
			\begin{align*}
				\Analyze(r) &= \bigcup_{1 \leq i \leq m} \set{(\emptyset, \Vars(a_i))} \cup \bigcup_{1 \leq i \leq n} \ProDep(l_i),
			\end{align*}
		\item
			for a conditional literal $l$ of form~$l_0 : l_1, \dots, l_n$, we let
			\begin{align*}
				\Analyze(l) &= \set{(\emptyset, \Vars(l_0))} \cup \bigcup_{1 \leq i \leq n} \ProDep(l_i),
			\end{align*}
		\item
			for an aggregate element $e$ of form~$\tuple{t}: l_1, \dots, l_n$, we let
			\begin{align*}
				\Analyze(e) &= \set{(\emptyset, \Vars(\tuple{t}))} \cup \bigcup_{1 \leq i \leq n} \ProDep(l_i).
			\end{align*}
	\end{itemize}
	Given a set~$\PD$ of pairs of sets of variables and a set of variables $V$, we let $\PD_{|V} = \set{(P \cap V, D \cap V) \mid (P, D) \in \PD}$.


	\subsection{Safety Definition}
	We define operator~$\CheckOp{\PD}$ parametrized with a set~$\PD$ of pairs of sets of variables
	applied to a set~$V$ of variables as
	\begin{align*}
	\CheckOp{\PD}(V) &= \bigcup_{(P,D) \in \PD, D \subseteq V} P.
	\end{align*}

	Let $r$ be a rule with global variables $G$.
	We say that a rule~$r$ is \emph{globally safe}
	if $G$ is the least fixed point of $\CheckOp{\PD}$ with $\PD=\Analyze{(r)}_{|G}$.
	Furthermore, we say that a conditional literal or aggregate element $e$ occurring in rule~$r$ is \emph{locally safe}
	if $L = \Vars(e) \setminus G$ is the least fixed point of $\CheckOp{\PD}$ with $\PD=\Analyze{(e)}_{|L}$.

	A rule is \emph{safe} if it is globally safe and all occurrences of conditional literals and aggregate elements in it are locally safe.

	\section{Other Examples}
	\begin{align*}
		dep(p(X;Y)) &= \set{(\Provide(X;Y), \emptyset), (\emptyset, \Depend(X;Y))} \\
		&= \set{(\Provide(X) \cap \Provide(Y), \emptyset), (\emptyset, (\Depend(X) \cup \Depend{Y}) \setminus (\Provide(X) \cap \Provide(Y)))} \\
		&= \set{(\set{X} \cap \set{Y}, \emptyset), (\emptyset, (\emptyset \cup \emptyset) \setminus (\set{X} \cap \set{Y}))} \\
		&= \set{(\emptyset, \emptyset), (\emptyset, \emptyset)}
	\end{align*}
	\begin{align*}
		dep(p(X;X+Y)) &= \set{(\Provide(X;X+Y), \emptyset), (\emptyset, \Depend(X;X+Y))} \\
		&= \set{(\Provide(X) \cap \Provide(X+Y), \emptyset), (\emptyset, (\Depend(X) \cup \Depend{X+Y}) \setminus (\Provide(X) \cap \Provide(X+Y)))} \\
		&= \set{(\set{X} \cap \emptyset, \emptyset), (\emptyset, (\emptyset \cup (\Vars(X+Y) \setminus \Provide(X+Y))) \setminus (\set{X} \cap \emptyset))} \\
		&= \set{(\emptyset, \emptyset), (\emptyset, (\set{X,Y} \setminus \emptyset) \setminus \emptyset)} \\
		&= \set{(\emptyset, \emptyset), (\emptyset, \set{X,Y})}
	\end{align*}
\end{document}

% vim: set noexpandtab:
