\documentclass{article}
\usepackage{bm}
\usepackage{breqn}
\usepackage{amsmath}

\title{Definition of Safety}
\author{Günther Wullaert}
\date{May 2022}

\newcommand{\pool}[1]{\boldsymbol{#1}}
\newcommand{\tuple}[1]{\dot{#1}}

\newcommand{\set}[1]{\{#1\}}
\newcommand{\dep}[2]{\{(#1), (#2)\}}
\newcommand{\provide}[2]{\{(\{#1\}, \{#2\})\}}
\newcommand{\provideM}[4]{\{(\{#1\}, \{#2\}), (\{#3\},\{#4\})\}}
\newcommand{\provideMM}[6]{\{(\{#1\}, \{#2\}), (\{#3\},\{#4\}), (\{#5\},\{#6\})\}}
\newcommand{\provideMMM}[8]{\{(\{#1\}, \{#2\}), (\{#3\},\{#4\}), (\{#5\},\{#6\}), (\{#7\},\{#8\})\}}

\begin{document}
	\maketitle
	
	\section{Language}
	\subsection{Term and Pools}
	
	We inductively define terms, tuples of terms, and pools as
	\begin{itemize}
		\item all numerals, symbolic constants, and variables are terms 
		\item $f(\pool{t})$ is a term, if $f$ is a symbolic constant and $\pool{t}$ is a pool
		\item ($t_1 \star t_2$) is a term, if $t_1$ and $t_2$ are terms and $\star$ is one of the symbols (+ - × / ..)
		\item $\langle \pool{t} \rangle$ is a term, if $\pool{t}$ is a pool, which can have a possibly empty set of terms.
		\item $t_1,...,t_n$ is a tuple of terms, if $n \ge 0$ and $t_i$ is a term.
  		\item $\tuple{t_1};...;\tuple{t_n}$ is a pool, if $n \ge 1$ and each $\tuple{t_i}$ is a tuple of terms. In particular, every tuple of terms is a pool.
	\end{itemize}

	\subsection{Constants}

	We inductively define a term to be $\mathit{constant}$ if:
	\begin{itemize}
		\item it is a numeral
		\item it has form $t \star u$ where t and u are $\mathit{constant}$ and $\star$ is one of the symbols (+ - × /)
	\end{itemize}

	\subsection{Atoms and Literals}
	
	An atom has form $p(\pool{t})$ where $p$ is a predicate symbol and $\pool{t}$ is a pool. 
	\\
	A literal is either an atom or atom preceded by not.

	\subsection{Comparisons}

	A comparison has form $t_1 \prec t_2$, where $t_1,t_2$ are terms and $\prec$ is on of the symbols ($\leq,\ge,<,>,\neq$)

	\subsection{Rules}

	A rule $r$ has the form
	\begin{equation}
		H_1 \lor ... \lor H_k  \leftarrow B_1 \land ... \land B_m 
	\end{equation}
	($k, m \ge 0$), where each $H_i$ is either a symbolic literal, literal or an comparison and each $B_j$ is either a symbolic symbolic literal, literal or an comparison.
	$H_1 \lor ... \lor H_k$ is called the head. $B_1 \land ... \land B_m$ is called the body.

	\section{Safety}
	We define the function $\mathit{provide()}$. The provide function returns a set of pairs. Each pair
	has the form $(p, d)$, where $p$ is a set of variables the statement provides if $d$ variables are provided.

	\subsection{Helper Functions}
	The $\mathit{vars(e)}$ function returns all variables in an expression.
	\\ For Example:
	\begin{equation}
		vars(a(X) = b(Y)) = \set{X,Y}
	\end{equation}

	\subsection{Terms}
	\subsubsection{Constants}
	For any numeral $n$ and symbolic constant $f$:
	\begin{equation}
		pt(n) = pt(f) = dt(n) = dt(f) = \emptyset
	\end{equation}

	\subsubsection{Variables}
	For any variable $X$:
	\begin{align}
		pt(X) &= \set{X} \\
		dt(X) &= \emptyset
	\end{align}

	\subsubsection{Tuples}
	For any tuple of terms $t_1,...,t_n$:
	\begin{align}
		pt(t_1,...,t_n) &= pt(t_1) \cup \dots \cup pt(t_n) \\
		dt(t_1,...,t_n) &= dt(t_1) \cup \dots \cup dt(t_n)
	\end{align}

	\subsubsection{Pools}
	For any pool of terms $\tuple{t_1};...;\tuple{t_n}$:
	\begin{align}
		pt(\tuple{t_1};...;\tuple{t_n}) &= pt(\tuple{t_1}) \cap \dots \cap pt(\tuple{t_n}) \\
		dt(\tuple{t_1};...;\tuple{t_n}) &= dt(\tuple{t_1}) \cup \dots \cup dt(\tuple{t_n})
	\end{align}

	\subsubsection{Terms}
	For a term of form $f(\pool{t})$, where $f$ a symbolic constant and $\pool{t}$ a pool:
	\begin{align}
		pt(f(\pool{t})) &= pt(\pool{t}) \\
		dt(f(\pool{t})) &= dt(\pool{t})
	\end{align}
	%%
	For a term of form $a \star b$, where $a,b$ are terms and one of them is a $\mathit{constant}$ and $\star$ is one of the symbols (+ - x) or $a$ and $b$ are both constant and $\star$ is one of the symbols (+ - × / ..):
	\begin{align}
		pt(a \star b) &= pt(b \star a) = pt(a) \cup pt(b) \\
		dt(a \star b) &= dt(b \star a) = dt(a) \cup dt(b)
	\end{align}
	%%
	Otherwise for a term of form $a \star b$:
	\begin{align}
		pt(a \star b) &= \emptyset \\
		pt(a \star b) &= vars(a \star b)
	\end{align}
	%%
	For a term of form $-t$, where $t$ is a term:
	\begin{align}
		pt(-t) &= pt(t) \\
		dt(-t) &= dt(t)
	\end{align}
	For a term of form $t*0$, where $t$ is a term:
	\begin{align}
		pt(t*0) &= pt(0*t) = \emptyset \\
		dt(t*0) &= dt(0*t) = \emptyset
	\end{align}

	\subsection{Atoms and Literals}
	\subsubsection{Atoms}
	For an atom of form $p(\pool{t})$, where $\pool{t}$ is a pool:
	\begin{equation}
		dep(p(\pool{t})) = \dep{pt(\pool{t}), \emptyset}{\emptyset, dt(\pool{t})}
	\end{equation}

	\subsubsection{Literals}
	For an literal of form $not \; a$, where $a$ is an atom:
	\begin{equation}
		dep(not \; a) = \set{(\emptyset, vars(a))}
	\end{equation}
	For an literal of form $a$, where $a$ is an atom:
	\begin{equation}
		dep(a) = dep(a)
	\end{equation}
	For an literal of form $not \; l$, where $l$ is another literal:
	\begin{equation}
		dep(not \; l) = \set{(\emptyset, vars(a))}
	\end{equation}

	\subsection{Comparisons}
	\subsubsection{Comparisons}
	For an comparison of form $a \prec b$, where $a$ and $b$ are terms and $\prec$ is one of the symbols ($\leq,\ge,<,>,\neq$):
	\begin{equation}
		dep(a \prec b) = \set{(\emptyset, vars(a \prec b))}
	\end{equation}
	For an comparison of form $a = b$, where $a$ and $b$ are terms:
	\begin{equation}
		dep(a = b) = \set{(pt(a), vars(b)), (pt(b), vars(a)), (\emptyset, dt(a) \cup dt(b))}
	\end{equation}

	\subsection{Rule}
	For a rule $r$ in the form of
	\begin{equation}
		H_1 \lor ... \lor H_k  \leftarrow B_1 \land ... \land B_m 
	\end{equation}
	The following holds:
	\begin{dmath}
		dep(r) = \set{(\emptyset, vars(H_1\vee ... \vee H_k))} \cup dep(B_1) \cup ... \cup dep(B_m)
	\end{dmath}

	\section{Extras}
	If 2 pairs share the same $p$, the $d$ can be merged. If 2 pairs share the same $d$, the $p$ can be merged.
	\\ For example:
	\begin{equation}
		\provideM{X}{}{Y}{} = \provide{X,Y}{}
	\end{equation}
	\begin{equation}
		\provideM{}{X}{}{Y} = \provide{}{X,Y}
	\end{equation}

	\section{Other Examples}
	\begin{dmath}
		dep(p(X,Y+Y)) = \dep{pt(X,Y+Y), \emptyset}{\emptyset, dt(X,Y+Y)}
		= \dep{pt(X) \cup pt(Y+Y), \emptyset}{\emptyset, dt(X) \cup dt(Y+Y)}
		= \dep{\set{X} \cup \emptyset, \emptyset}{\emptyset, \emptyset \cup vars(Y+Y)}
		= \dep{\set{X}, \emptyset}{\emptyset, \set{Y}}
	\end{dmath}
	\begin{dmath}
		dep(a(Y) \leftarrow a(X), X=Y) = \set{(\emptyset, vars(a(Y)))} \cup dep(a(X)) \cup dep(X\mathit{=}Y)
		= \set{(\emptyset, \set{Y})} \cup \dep{pt(a(X)), \emptyset}{\emptyset, dt(a(X))} \cup \set{(pt(X), vars(Y)), (pt(Y), vars(X)), (\emptyset, dt(X) \cup dt(Y))}
		= \set{(\emptyset, \set{Y}), (pt(X), \emptyset), (\emptyset, dt(X)), (\set{X}, \set{Y}), (\set{Y}, \set{X}), (\emptyset, \emptyset \cup \emptyset)}
		= \set{(\emptyset, \set{Y}), (\set{X}, \emptyset), (\emptyset, \emptyset), (\set{X}, \set{Y}), (\set{Y}, \set{X}), (\emptyset, \emptyset)}
		= \set{(\emptyset, \set{Y}), (\set{X}, \emptyset), (\emptyset, \emptyset), (\set{X}, \set{Y}), (\set{Y}, \set{X})}
	\end{dmath}
\end{document}